\documentclass[11pt]{article}
\title{\textbf{CS 361 Spring 2018\\Homework 5}}
\author{Nathaniel Murphy}
\date{}

\usepackage{a4wide}
\usepackage{amsfonts}
\usepackage{amsmath}
\usepackage{amsthm}
\usepackage{graphicx}

\begin{document}
\maketitle

\section*{5.9}
\subsection*{(a)}
In this case, we can completely disregard $p$, except for the fact that we know that both flips have the same probability $p$ of being heads. To terminate this algorithm, we notice that one of the flips will be heads and one will be tails. Since the coin has the same probability for both flips and the flips are independent, we may say that the first flip being heads has the same probability as the second flip being heads, hence, $\frac{1}{2}$.
\subsection*{(b)}
Let $X$ be the random variable of the number of pairs of flips you complete before terminating the algorithm.
\[P(X=1)=2p(1-p)\]
\[P(X=2)=(p^2+(1-p)^2) \cdot 2(p(1-p))\]
\[P(X=3)=(p^2+(1-p)^2)^2\cdot 2(p(1-p))\]
We see that $X$ follows a geometric distribution because
\[P(X=x)=(p^2+(1-p)^2)^{x-1}\cdot 2(p(1-p))\]
Which follows because we notice that
\[p^2+2(p(1+p))+(1-p)^2=(p+1-p)^2=1\]
From the properties of the geometric distribution, we see that
\[\mathbb{E}[X]=\frac{1}{2(p(1-p))}\]
And since this is the expectation for the number of pairs of flips, we can write
\[\mathbb{E}[\text{number of flips}]=\mathbb{E}[2X]=2\mathbb{E}[X]=\frac{2}{2(p(1-p))}=\frac{1}{p(1-p)}\]

\section*{5.17}
Let $X$ be the random variable of the number of women on the flight. \\ \\
We want $P(X\geq1)=1-P(X=0)$.
\[P(X=0)=\left(\frac{1}{2}\right)^6=\frac{1}{64}\]
\[P(X\geq1)=1-\frac{1}{64}=\frac{63}{64}\approx0.9844\]
The probability that the plane flies is approximately 0.9844.

\section*{5.25}
\[f(x)=\frac{1}{\sqrt{2\pi}}e^{-\frac{x^2}{2}}=\frac{1}{\sqrt{2\pi}}\cdot\frac{1}{e^{\frac{x^2}{2}}}\]
\subsection*{(a)}
We notice that $\frac{1}{e^{\frac{x^2}{2}}}$ is always positive because $e^x>0\hspace{1mm}\forall\hspace{1mm}x\in\mathbb{R}$. This implies that
\[\frac{1}{\sqrt{2\pi}}e^{-\frac{x^2}{2}}>0\hspace{1mm}\forall\hspace{1mm}x\in\mathbb{R}\]
\subsection*{(b)}
\[\int_{-\infty}^{\infty}f(x)dx=\int_{-\infty}^{\infty}\left(\frac{1}{\sqrt{2\pi}}\right)e^{-\frac{x^2}{2}}dx\]
Let us integrate using polar coordinates. \\ \\
Note that
\[\left(\int_{-\infty}^{\infty}e^{-\frac{x^2}{2}}\right)^2=\int_{-\infty}^{\infty}e^{-\frac{x^2}{2}}dx\int_{-\infty}^{\infty}e^{-\frac{y^2}{2}}dy=\int_{-\infty}^{\infty}\int_{-\infty}^{\infty}e^{-\frac{x^2+y^2}{2}}dxdy=\int\int_{R^2}e^{-\frac{x^2+y^2}{2}}d(x,y)\]
Knowing that $x=r\text{cos}\theta$ and $y=r\text{sin}\theta,\hspace{1mm}x^2+y^2=r^2(\text{sin}^2\theta+\text{cos}^2\theta)=r^2$.
\[=\int_0^{2\pi}\int_0^{\infty}e^{-\frac{r^2}{2}}rdrd\theta=2\pi\int_0^{\infty}re^{-\frac{r^2}{2}}\]
Let $u=\frac{r^2}{2},\hspace{1mm}du=r$
\[=2\pi\int_0^{\infty}e^{-u}du=-2\pi\left[e^{-\frac{r^2}{2}}\right]_0^{\infty}=-2\pi\left[e^{-\infty}-e^0\right]=2\pi\]
It follows that 
\[\left(\int_{-\infty}^{\infty}e^{-\frac{x^2}{2}}\right)^2=2\pi\Rightarrow\int_{-\infty}^{\infty}e^{-\frac{x^2}{2}}=\sqrt{2\pi}\]
This implies that
\[\int_{-\infty}^{\infty}\left(\frac{1}{\sqrt{2\pi}}\right)e^{-\frac{x^2}{2}}dx=\left(\frac{1}{\sqrt{2\pi}}\right)\int_{-\infty}^{\infty}e^{-\frac{x^2}{2}}dx-\left(\frac{1}{\sqrt{2\pi}}\right)\left(\sqrt{2\pi}\right)=1\]
\subsection*{(c)}
\[f(-x)=\left(\frac{1}{\sqrt{2\pi}}\right)e^{-\frac{x^2}{2}}=\left(\frac{1}{\sqrt{2\pi}}\right)e^{-\frac{x^2}{2}}=f(x)\]
\[f(-x)=f(x)\Rightarrow\int_0^{\infty}f(x)dx=\int_{-\infty}^0f(x)dx\]
Furthermore,
\[f(-x)-f(x)=0\Rightarrow xf(-x)-xf(x)=xf(x)+(-xf(x))=0,\hspace{1mm}\forall\hspace{1mm}x\in\mathbb{R}\]
It follows that
\[\int_0^{\infty}xf(x)dx=-\int_0^{\infty}-xf(x)dx=-\int_0^{\infty}-xf(-x)dx=-\int_{-\infty}^0xf(x)dx\]
\[\int_{-\infty}^{\infty}xf(x)dx=\int_0^{\infty}xf(x)dx+\int_{-\infty}^0xf(x)dx=-\int_{-\infty}^0xf(x)dx+\int_{-\infty}^0xf(x)dx=0\]
\subsection*{(d)}
Let $s=x-\mu\Rightarrow x=s+\mu,\hspace{1mm}ds=dx$.
\[\int_{-\infty}^{\infty}xf(x-\mu)dx=\int_{-\infty}^{\infty}(s+\mu)f(s)ds=\int_{-\infty}^{\infty}sf(s)+\mu f(s)ds=\int_{-\infty}^{\infty}sf(s)ds+\mu\int_{-\infty}^{\infty}f(s)ds=0+\mu=\mu\]
This comes from the fact that
\[\int_{-\infty}^{\infty}f(s)ds=1\text{ and }\int_{-\infty}^{\infty}sf(s)ds=0\]
\subsection*{(e)}
To integrate this, we will be using the gamma function. We will be using the following information:
\begin{itemize}
	\item $\Gamma(\alpha)=\int_0^{\infty}x^{\alpha-1}e^{-x}dx$
	\item $\Gamma(\frac{3}{2})=\frac{\sqrt{\pi}}{2}$
\end{itemize}
Let $u=\frac{x^2}{2},\hspace{1mm}du=xdx$. $u=\frac{x^2}{2}\Rightarrow x=\sqrt{2u}$.
\[\int_{-\infty}^{\infty}x^2e^{-\frac{x^2}{2}}dx=2\int_0^{\infty}x^2e^{-\frac{x^2}{2}}dx=2\int_0^{\infty}xe^{-\frac{x^2}{2}}xdx=2\int_0^{\infty}\sqrt{2u}e^{-u}du=2\sqrt{2}\int_0^{\infty}u^{\frac{1}{2}}e^{-u}du\]
\[=2\sqrt{2}\cdot\Gamma\left(\frac{3}{2}\right)2\sqrt{2}\left(\frac{\sqrt{\pi}}{2}\right)=\sqrt{2\pi}\]
It follows that
\[\int_{-\infty}^{\infty}x^2\left(\frac{1}{\sqrt{2\pi}}e^{-\frac{x^2}{2}}\right)dx=\frac{1}{\sqrt{2\pi}}\int_{-\infty}^{\infty}x^2e^{-\frac{x^2}{2}}dx=\frac{1}{\sqrt{2\pi}}\cdot\sqrt{2\pi}=1\]
\clearpage
\section*{5.26}
\[g(x)=e^{-\frac{(x-\mu)^2}{2\sigma^2}}\]
Let $u=\frac{x-\mu}{\sigma}$, $du=\frac{1}{\sigma}$. Notice that
\[\int_{-\infty}^{\infty}\frac{1}{\sqrt{2\pi}}e^{-\frac{x^2}{2}}dx=\frac{1}{\sqrt{2\pi}}\int_{-\infty}^{\infty}e^{-\frac{x^2}{2}}dx=1\Rightarrow\int_{-\infty}^{\infty}e^{-\frac{x^2}{2}}dx=\sqrt{2\pi}\]
\[\int_{-\infty}^{\infty}e^{-\frac{(x-\mu)^2}{2\sigma^2}}dx=\sigma\int_{-\infty}^{\infty}e^{-\frac{x^2}{2}}du=\sigma\sqrt{2\pi}\]

\section*{5.28}
\subsection*{(a)}
$N=1000000$
\[a=\frac{49500-500000}{\sqrt{1000000\cdot0.5\cdot0.5}}=-\frac{450500}{500}=901\]
\[b=\frac{50500-5000000}{\sqrt{1000000\cdot0.5\cdot0.5}}=\frac{449500}{500}=899\]
\[P(h\in[49500,50500])=\int_{899}^{901}\frac{1}{\sqrt{2\pi}}e^{-\frac{x^2}{2}}dx\approx0\]
\subsection*{(b)}
$N=10000$
\[a=\frac{9000-5000}{\sqrt{10000\cdot0.5\cdot0.5}}=\frac{4000}{50}=80\]
\[P(h>9000)=\int_{80}^{\infty}\frac{1}{\sqrt{2\pi}}e^{-\frac{x^2}{2}}dx\approx0\]
\subsection*{(c)}
$N=100$
\[b=\frac{40-50}{\sqrt{100\cdot0.5\cdot0.5}}=-\frac{10}{5}=-2\]
\[a=\frac{60-50}{\sqrt{100\cdot0.5\cdot0.5}}=\frac{10}{5}=2\]
\[P(\{h>60\}\cup\{h<40\})=\int_{-\infty}^{-2}\frac{1}{\sqrt{2\pi}}e^{-\frac{x^2}{2}}dx+\int_{2}^{\infty}\frac{1}{\sqrt{2\pi}}e^{-\frac{x^2}{2}}dx\approx0.02275+0.02275=0.0455\]
\end{document}