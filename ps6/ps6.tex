\documentclass[11pt]{article}
\title{\textbf{CS 361 Spring 2018\\Homework 6}}
\author{Nathaniel Murphy (njmurph3)}
\date{}

\usepackage{a4wide}
\usepackage{amsfonts}
\usepackage{amsmath}
\usepackage{amsthm}
\usepackage{graphicx}

\begin{document}
\maketitle
\ \\
**Note that in this problem set we will be using $X$ to denote the population of what we are sampling. \\
\section*{6.2}
\subsection*{(a)}
The best estimate for population mean of the weight of the mice is mean of the sample of the mice.
\[\text{popmean}(\{X\})=\text{mean}(\{x\})=\frac{1}{N}\sum_{i=1}^Nx_i=\frac{199}{10}=\textbf{19.9}\text{ grams}\]
\subsection*{(b)}
\[\text{stderr}(\{x\})=\frac{\text{stdunbiased}(\{x\})}{\sqrt{N}},\text{ stdunbiased}(\{x\})=\sqrt{\frac{\sum_{i=1}^N(x_i-\text{mean}(\{x\}))^2}{N-1}}\]
\[\text{stdunbiased}(\{x\})=\sqrt{\frac{\sum_{i=1}^N(x_i-\text{mean}(\{x\}))^2}{N-1}}=\sqrt{\frac{110.9}{9}}=3.51\]
\[\text{stderr}(\{x\})=\frac{\text{stdunbiased}(\{x\})}{\sqrt{N}}=\frac{3.51}{\sqrt{10}}=\textbf{1.11}\text{ grams}\]
\subsection*{(c)}
Assuming that stdunbiased$(\{x\})$ parameter does not change as the sample size $N$ changes, we have the equation:
\[\text{std}(\{x\})=\frac{\text{stdunbiased}(\{x\})}{\sqrt{N}}=0.1\]
\[\frac{\text{stdunbiased}(\{x\})}{0.1}=\sqrt{N}\]
\[\left(\frac{\text{stdunbiased}(\{x\})}{0.1}\right)^2=N\]
\[\left(\frac{3.51}{0.1}\right)^2=\textbf{1233}\text{ mice}\]
\clearpage
\section*{6.3}
Let us assume that we have the relation yellow $\rightarrow$ 0 and blue $\rightarrow$ 1.
\subsection*{(a)}
\[\text{stdunbiased}(\{x\})=\sqrt{\frac{\sum_{i=1}^N(x_i-\text{mean}(\{x\}))^2}{N-1}}=\sqrt{\frac{7(-0.3)^2+3(0.7)^2}{9}}=0.483\]
\[\text{stderr}(\{x\})=\frac{\text{stdunbiased}(\{x\})}{\sqrt{N}}=\frac{0.483}{\sqrt{10}}=\textbf{0.153}\]
\subsection*{(b)}
Assuming that stdunbiased$(\{x\})$ does not change as $N$ changes, we have the algorithm:
\[\text{stderr}(\{x\})=\frac{\text{stdunbiased}(\{x\})}{\sqrt{N}}=0.05\]
\[\frac{\text{stdunbiased}(\{x\})}{0.05}=\sqrt{N}\]
\[\left(\frac{\text{stdunbiased}(\{x\})}{0.05}\right)^2=N\]
\[\left(\frac{0.483}{0.05}\right)^2=\textbf{94}\text{ draws}\]
\section*{6.4}
Given $N=40$, stdunbiased$(\{x\})=75$ grams, let us compute stderr$(\{x\})$.
\[\text{stderr}(\{x\})=\frac{\text{stdunbiased}(\{x\})}{\sqrt{N}}=\frac{75}{\sqrt{40}}=11.859\text{ grams}\]
\subsection*{(a)}
A 68\% confidence interval for popmean$(\{x\})$ is given by the following formula:
\[\text{mean}(\{x\})-\text{stderr}(\{x\})\leq\text{popmean}(\{x\})\leq\text{mean}(\{x\})+\text{stderr}(\{x\})\]
\[340-11.859\leq\text{popmean}(\{x\})\leq340+11.859\]
\[328.141\leq\text{popmean}(\{x\})\leq351.859\]
\subsection*{(b)}
A 99\% confidence interval for popmean$(\{x\})$ is given by the following formula:
\[\text{mean}(\{x\})-3\cdot\text{stderr}(\{x\})\leq\text{popmean}(\{x\})\leq\text{mean}(\{x\})+3\cdot\text{stderr}(\{x\})\]
\[340-3(11.859)\leq\text{popmean}(\{x\})\leq340+3(11.859)\]
\[304.123\leq\text{popmean}(\{x\})\leq375.577\]
\clearpage
\section*{6.5}
We observe that z-values for 0.8 and 0.95 are 0.845 adn 1.96, respectively. \\ \\
The sample dataset is identical to that in problem 6.2, so we may say that mean$(\{x\})=19.9$ grams and stderr$(\{x\})=1.11$ grams
\subsection*{(a)}
The 80\% confidence interval for popmean$(\{X\})$:
\[\text{mean}(\{x\})-0.845(\text{stderr}(\{x\}))\leq\text{popmean}(\{x\})\leq\text{mean}(\{x\})+0.845(\text{stderr}(\{x\}))\]
\[19.9-0.845(1.11)\leq\text{popmean}(\{x\})\leq19.9+0.845(1.11)\]
\[18.962\leq\text{popmean}(\{x\})\leq20.838\]
\subsection*{(b)}
The 95\% confidence interval for popmean$(\{x\})$:
\[\text{mean}(\{x\})-1.96(\text{stderr}(\{x\}))\leq\text{popmean}(\{x\})\leq\text{mean}(\{x\})+1.96(\text{stderr}(\{x\}))\]
\[19.9-1.96(1.11)\leq\text{popmean}(\{x\})\leq19.9+1.96(1.11)\]
\[17.724\leq\text{popmean}(\{x\})\leq22.076\]
\section*{6.6}
Let $X$ be the random variable of a child that is born in Carcelle-le-Grignon. Let $x$ denote the random sample. Assuming that female $\rightarrow$ 1 and male $\rightarrow$ 0, we have that 
\begin{itemize}
	\item $P(x=0)=\frac{983}{2009}=0.4892$
	\item $P(x=1)=\frac{1026}{2009}=0.511$
\end{itemize}
\[\text{stdunbiased}(\{x\})=\sqrt{\frac{\sum_{i=1}^N(x_i-\text{mean}(\{x\}))^2}{N-1}}=\sqrt{\frac{983(0.511)^2+1026(0.4892)^2}{2008}}=0.5005\]
\[\text{stderr}(\{x\})=\frac{\text{stdunbiased}(\{x\})}{\sqrt{N}}=\frac{0.5005}{\sqrt{2009}}=0.0112\]
\subsection*{(a)}
The 99\% confidence interval for $P(X=1)$:
\[P(x=1)-3\cdot\text{stderr}(\{x\})\leq P(X=1)\leq P(x=1)+3\cdot\text{stderr}(\{x\})\]
\[0.511-3(0.0112)\leq P(X=1)\leq 0.0511+3(0.0112)\]
\[0.4771\leq P(X=1)\leq 0.5443\]
\subsection*{(b)}
The 99\% confidence interval for $P(X=0)$:
\[P(x=0)-3\cdot\text{stderr}(\{x\})\leq P(X=0)\leq P(x=0)+3\cdot\text{stderr}(\{x\})\]
\[0.4892-3(0.0112)\leq P(X=0)\leq 0.4892+3(0.0112)\]
\[0.4557\leq P(X=0)\leq 0.5229\]
\subsection*{(c)}
These intervals do overlap, so this implies that the probabilities could be equal of having a boy or having a girl.
\end{document}